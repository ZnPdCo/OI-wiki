% Modified from https://tex.stackexchange.com/a/261567.
\documentclass{standalone}
\usepackage[active,tightpage]{preview}
\usepackage{amsmath}
\usepackage{pgfplots}笑了
\usepgfplotslibrary{fillbetween}
\usetikzlibrary{arrows.meta,intersections}

\usepackage{anyfontsize}
\renewcommand{\normalsize}{\fontsize{14pt}{16pt}\selectfont}

\DeclareMathOperator{\dom}{dom}

\pgfmathdeclarefunction{func}{1}{%
  \pgfmathparse{max(6*(#1-4)*(#1-4),17)+3*(#1-4)}%
}

\pgfplotsset{compat=1.16}
\makeatletter
\pgfplotsset{
    mark max/.style={
        point meta rel=per plot,
        visualization depends on={x \as \xvalue},
        scatter/@pre marker code/.code={%
        \ifx\pgfplotspointmeta\pgfplots@metamax
            \def\markopts{mark=none}%
            \coordinate (maximum);
        \fi
            \def\markopts{mark=none}
            \expandafter\scope\expandafter[\markopts]
        },%
        scatter/@post marker code/.code={%
            \endscope
        },
        scatter
    }
}

% Syntax
% \DrawEpigraph[<additional options>]{<min domain x>}{<max domain x>}{<shift on the left>}{<shift on the right>}
\newcommand\DrawEpigraph[6][draw=white,top color=red!05,bottom color=red!45]{
  \coordinate (plot-left) at ([yshift=#4]axis cs:#2,\pgfplots@metamax);
  \coordinate (plot-right) at ([yshift=#5]axis cs:#3,\pgfplots@metamax);
  \path[name path=diagonal,draw=none] (plot-left) -- (plot-right);
  \addplot[#1] fill between[of=#6 and diagonal];
}
\makeatother

\PreviewEnvironment{tikzpicture}

\begin{document}        

\begin{tikzpicture}
\begin{axis}[
  axis lines=middle,
  xlabel={$x$},
  ylabel={$y$},
  xtick={\empty},
  ytick={\empty},
  domain=0.5:7,
  ymin=0,
  ymax=64,
  xmin=0,
  xmax=7.5,
  clip=false,
  axis on top,
  scale=2
] 
% The curve
\addplot [mark max,black,thick,name path=B,samples=100] plot {func(x)};
% The Epigraph
\DrawEpigraph{0.5}{7}{0}{0}{B}
% Lines and labels
\draw[dashed] 
    (axis cs:1,0) node[below] {$x$} -- (axis cs:1,{func(1)})
    node[
        draw,fill,circle,inner sep=-1.5pt,
        label=180:$f(x)$
    ] {};
\draw[dashed] 
    (axis cs:6,0) node[below] {$y$} -- (axis cs:6,{func(6)})
    node[
        draw,fill,circle,inner sep=-1.5pt,
        label=0:$f(y)$
    ] {};
\draw[dashed] 
    (axis cs:3,0) node[below] {$\alpha x+(1-\alpha)y$} 
    -- (axis cs:3,{(func(1)*3+func(6)*2)/5})
    node[
        draw,fill,circle,inner sep=-1.5pt,
        label=45:$\alpha f(x)+(1-\alpha)f(y)$
    ] {};
\draw[dashed,thick] (axis cs:1,{func(1)}) -- (axis cs:6,{func(6)});
\node[below left] at (axis cs:0,0) {$O$};
\node[draw,fill,circle,inner sep=-1.5pt,label=-45:$f\left(\alpha x+(1-\alpha)y\right)$]
    at (axis cs:3,{func(3)}) {}; 
\end{axis}
\end{tikzpicture}

\end{document}